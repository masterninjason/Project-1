\documentclass{uioposter}


\usepackage[figwidth = 0.98\linewidth]{todonotes}  % Dummy image (and more!)
\usepackage[absolute, overlay]{textpos}% Figure placement
\setlength{\TPHorizModule}{\paperwidth}
\setlength{\TPVertModule}{\paperheight}


\title{Project \#1}
\author
{%
    AKA. 3 way Duel
    \and
    AKA. Source Duels
    \and
    AKA. Ninjason's unnamed project
}
%% Optional:
\institute
{
    3 Players
    \and
    5-10 minutes
    \and
    ages 10?+ idk
}
% Or:
%\institute{Contact information}


%% Remove footline:
\setbeamertemplate{footline}{}


\begin{document}
\begin{frame}
\begin{columns}[onlytextwidth]


\begin{column}{0.5\textwidth - 1.5cm}

    \begin{block}{Terminology}
        To get started, lets go over some basic terminology that may be used in describing characters abilities
        \vspace{2ex}
        \begin{block}{Place / Move}
            Moving a weapon is when it follows a path of empty spaced, if a space it would need to travel through has a weapon, that would block it from attacking any spaces further down the chain, this also applies to characters, you cannot attack the second character in a weapons movement if it must pass through a first.
            \\
            Placing a weapon can be done regardless of any clear path.
        \end{block}
        \vspace{-3ex}
        \begin{block}{Swing}
            swinging a weapon is when it moves a number of spaces remaining adjacent to you, for example, a swing of 3 spaces can reach exactly the opposite side of your character
        \end{block}
        \vspace{-3ex}
        \begin{block}{Diagonal}
            A Diagonal space is a space that is not directly adjacent, but adjacent to 2 of the adjacent spaces... a space that is 2 spaces away, but not in a straight line
        \end{block}
    \end{block}
    
    \begin{alertblock}{Universal Moves}
        Reminder: all characters have access to the basic move and jump unless otherwise stated
        \\
        \\
        Basic Move: move 1 space, then place your weapon in any adjacent space
        \\
        \\
        Jump: leave you weapon in the same space, and move your character to the opposite side of it
    \end{alertblock}

    \begin{alertblock}{Separated Weapons}
        if your weapon is not adjacent to you, unless otherwise stated by a character, you cannot us it, it stays in whichever space it lands until you move adjacent to it again\\
        If an opponents weapon is separated, that space is considered occupied like normal, except for moving directly into with a basic move, in which case move the weapon to an adjacent empty space (owners choice)
    \end{alertblock}

    
\end{column}


\begin{column}{0.5\textwidth - 1.5cm}

    \begin{exampleblock}{Character: Short Sword}
        Attack staying in the same spot, knocking out a player by moving your weapon in their space, you may "swing" your weapon upto 2 spaces, you may also attack using the jump ability, knocking out an opponent if they are where you land on a jump.
    \end{exampleblock}

    \begin{exampleblock}{Character: Boomerang}
        Attack moving your weapon around all the adjacent hexes of one of the hexes adjacent to both you and your weapon. attacking the first opponent it lands on
    \end{exampleblock}

    \begin{exampleblock}{Character: Hammer}
        Attack placing your weapon in one of the 3 adjacent spaces opposite your weapon.\\
        Also may attack moving into the space containing your weapon, then placing your weapon in the next space in a straight line.
    \end{exampleblock}

    \begin{exampleblock}{Character: Dagger}
        Attack by jumping over an opponent, placing your weapon in their space.\\
        you may also jump over opponent's weapons, leaving your weapon behind if it was adjacent to you.
    \end{exampleblock}

    \begin{exampleblock}{Character: Spear}
        You may attack with a Jump, or jumping, then placing your weapon in the next space in the same direction. (this is still a movement and does not require to be a successful attack)\\
        you may also attack by simply moving into the space containing your weapon, then placing your weapon in the next space in a straight line or diagonal to you from your starting position.
    \end{exampleblock}

    \begin{exampleblock}{Character: Lance}
        Attack moving your weapon and character in the same direction, where your weapon lands on an opponent.
    \end{exampleblock}

\end{column}


\end{columns}


\end{frame}
\end{document}
